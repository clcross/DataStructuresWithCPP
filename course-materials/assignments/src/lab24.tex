\documentclass[12pt]{article}
\usepackage{listings}
\usepackage{color}
\textwidth=7in
\textheight=9.5in
\topmargin=-1in
\headheight=0in
\headsep=.5in
\hoffset  -.85in

\definecolor{mygray}{rgb}{0.4,0.4,0.4}
\definecolor{mygreen}{rgb}{0,0.8,0.6}
\definecolor{myorange}{rgb}{1.0,0.4,0}

\lstset{
basicstyle = \ttfamily,columns=fullflexible,
commentstyle=\color{mygray},
frame=single,
numbers=left,
numbersep=5pt,
numberstyle=\tiny\color{mygray},
keywordstyle=\color{mygreen},
showspaces=false,
showstringspaces=false,
stringstyle=\color{myorange},
tabsize=2
}

\pagestyle{empty}

\renewcommand{\thefootnote}{\fnsymbol{footnote}}

\begin{document}

\begin{center}
{\bf lab 24 Quad Trees}

\end{center}

\setlength{\unitlength}{1in}

\begin{picture}(6,.1) 
\put(0,0) {\line(1,0){6.25}}         
\end{picture}

\renewcommand{\arraystretch}{2}
\setlength{\tabcolsep}{6pt} % General space between cols (6pt standard)
\renewcommand{\arraystretch}{.5} % General space between rows (1 standard)

\vskip.15in
\noindent\textbf{Instructions:} In this lab implement these basics of a QuadTree.

Implement the following class:
\begin{lstlisting}[language=C++]{Name=test2}
#ifndef QUADTREE_H__
#define QUADTREE_H__

#include <stdlib.h>
#include <vector>
#include <list>
#include <set>
#include <queue>

class QuadTree {
    private:
        /* Class to begin filling out...*/
    public:
        /* Initialize an empty quadtree. */
        QuadTree(float width, float height);

        /* add a point to the quadtree. */
        bool add(float x, float y);

        /* remove a point from the QuadTree.
         * Remember to remove empty QuadTree nodes, or your tree will 
         * use up too much memory when doing the add/remove test!
         */
        bool remove(float x, float y);

        /* returns if the quad tree has a point (x, y) */
        bool contains(float x, float y);

        /* return the number of points in the box (sx, sy) -> (ex, ey) 
         * You may assume that sx < ex and sy < ey!
         */
        int countInRange(float sx, float sy, float ex, float ey);

        void print();
};

#include "quadtree.cpp"

#endif

\end{lstlisting}

\vskip.15in
\noindent\textbf{Write some test cases:} \\
Create some test cases, using cxxtestgen, that you believe would cover all aspects of your code.

\vskip.15in
\noindent\textbf{Memory Management:} \\
Now that are using new, we must ensure that there is a corresponding delete to free the memory.  Ensure there are no memory leaks in your code!  Please run Valgrind on your tests to ensure no memory leaks!
\vskip.15in

\vskip.15in
\noindent\textbf{How to turn in:} \\
Turn in via GitHub.  Ensure the file(s) are in your directory and then:
\begin{itemize}
\item \$ git add $<$files$>$
\item \$ git commit 
\item \$ git push
\end{itemize}

\vskip.15in
\noindent\textbf{Due Date:}
December 04, 2019 2359

\vskip.15in
\noindent\textbf{Teamwork:} No teamwork, your work must be your own.

\end{document}
